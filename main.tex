\documentclass{scrbook} % KOMA-Script book class

%---------------------------------------------------------
% Encoding, language, quotes
%---------------------------------------------------------
\usepackage[utf8]{inputenc}
\usepackage[ngerman]{babel}       % German language & hyphenation
\usepackage{csquotes}             % Recommended for quotes with babel/polyglossia

%---------------------------------------------------------
% Graphics & PDF
%---------------------------------------------------------
\usepackage{graphicx}
\usepackage{pdfpages}             % For including PDF pages
\usepackage{caption}              % For \captionabove
\usepackage{subcaption}

%---------------------------------------------------------
% Math, units, spacing, etc.
%---------------------------------------------------------
\usepackage{siunitx}
\usepackage{setspace}
\usepackage{textgreek}

%---------------------------------------------------------
% Other packages
%---------------------------------------------------------
\usepackage{ifthen}
\usepackage{acronym}
\PassOptionsToPackage{hyphens}{url}% Tells url/hyperref to allow breaks at hyphens
\usepackage[breaklinks=true]{hyperref} 
\usepackage{array}                % Better table formatting
\usepackage{enumitem}             % Control over itemize environment
\usepackage{nomencl}

%---------------------------------------------------------
% Bibliography (biblatex with Biber)
%---------------------------------------------------------
\usepackage[backend=biber, style=ieee]{biblatex}  
\addbibresource{literatur.bib}  

%---------------------------------------------------------
% Custom placeholders
%---------------------------------------------------------
\newcommand{\titel}{Analyse und Implementierung eines mobilen Warnsystems zur Minimierung von Abbiegeunfällen zwischen Lkw und Fußgänger:innen}
\newcommand{\untertitel}{}
\newcommand{\arbeit}{Studienarbeit T3100}
\newcommand{\studiengang}{Elektrotechnik}
\newcommand{\studienrichtung}{Fahrzeugelektronik}
\newcommand{\autor}{Luka Tadic}
\newcommand{\abgabe}{13.01.2025}
\newcommand{\bearbeitungszeitraum}{09.10.2024 - 13.01.2025}
\newcommand{\matrikelnr}{5726700}
\newcommand{\kurs}{TFE22-1, TFE22-2}
\newcommand{\firma}{}
\newcommand{\betreuerfirma}{Prof. Dr. Ing. Tobias Frank}
\newcommand{\gutachterdhbw}{Prof. Dr. Ing. Tobias Frank}
\newcommand{\jahr}{2025} % or \the\year if you want the current year

%---------------------------------------------------------
% Document Start
%---------------------------------------------------------
\begin{document}

%---------------------------------------------------------
% Title Page
%---------------------------------------------------------
\thispagestyle{plain}
\hypersetup{pageanchor=false}

\begin{titlepage}
\enlargethispage{4.0cm}
\sffamily  % Serifenlose Grundschrift für die Titelseite

\parbox{0.5\linewidth}{
    \begin{flushleft}
        % Optional: Insert a company logo here
    \end{flushleft}
}
\parbox{0.5\linewidth}{
    \begin{flushright}
        \includegraphics[width=0.4\linewidth]{images/DHBW_d_R_FN_46mm_4c}\\[5ex]
    \end{flushright}
}

\begin{center}

{\fontsize{20.74pt}{24pt}\selectfont
\textbf{\titel}\\[1.5ex]}

{\fontsize{14pt}{17pt}\selectfont
\textbf{\untertitel}\\[5ex]}

{\fontsize{17pt}{20pt}\selectfont
\textbf{\arbeit}\\[2ex]}

{\fontsize{14pt}{17pt}\selectfont
Studiengang \studiengang\\[2ex]}

{\fontsize{12pt}{14pt}\selectfont
Studienrichtung \studienrichtung\\[1ex]
Duale Hochschule Baden-Württemberg Ravensburg, Campus Friedrichshafen\\[5ex]
von\\[1ex]
\autor\\[15ex]}

\end{center}

\begin{center}
{\fontsize{12pt}{14pt}\selectfont
\begin{tabular}{ll}
Abgabedatum:                    & \quad \abgabe \\
Bearbeitungszeitraum:           & \quad \bearbeitungszeitraum \\
Matrikelnummer:                 & \quad \matrikelnr \\
Kurs:                           & \quad \kurs \\
Dualer Partner:                 & \quad \firma \\ % entfällt bei Studienarbeit
Betreuerin / Betreuer:          & \quad \betreuerfirma \\ % Betreuer
Gutachterin / Gutachter:        & \quad \gutachterdhbw \\ [2ex]
\end{tabular}
}
\end{center}

% Uncomment if needed for copyright:
% \begin{flushleft}
% {\fontsize{11pt}{13pt}\selectfont
% Copyrightvermerk:\\
% Dieses Werk einschließlich seiner Teile ist \textbf{urheberrechtlich geschützt}.
% Jede Verwertung außerhalb der engen Grenzen des Urheberrechtgesetzes 
% ist ohne Zustimmung des Autors unzulässig und strafbar.
% }
% \end{flushleft}
% \begin{flushright}
% {\fontsize{11pt}{13pt}\selectfont \copyright{} \jahr }
% \end{flushright}

\end{titlepage}

\ifthenelse{\boolean{@twoside}}{%
    \cleardoublepage
}{%
    \clearpage
}%

\hypersetup{pageanchor=true}

%---------------------------------------------------------
% (Optional) Sperrvermerk
%---------------------------------------------------------
% \chapter*{Sperrvermerk}
% ... text ...

%---------------------------------------------------------
% Erklärung
%---------------------------------------------------------
\chapter*{Erklärung} % Nicht im Inhaltsverzeichnis auftauchen

gemäß Ziffer 1.1.14 der Anlage 1 zu §§ 3, 4 und 5  
der Studien- und Prüfungsordnung für die Bachelorstudiengänge im Studienbereich Technik  
der Dualen Hochschule Baden-Württemberg vom 29.09.2017 in der Fassung vom 24.07.2023.

Ich versichere hiermit, dass ich meine \arbeit\ mit dem Thema:

\begin{quote}
    \textit{\titel}
\end{quote}

selbstständig verfasst und keine anderen als die angegebenen Quellen und Hilfsmittel benutzt habe.  
Ich versichere zudem, dass die eingereichte elektronische Fassung mit der gedruckten Fassung übereinstimmt.\\[6ex]

Friedrichshafen, den \today \\[1ex]
\rule[-0.2cm]{5cm}{0.5pt} \\
\autor \\[10ex]

\rmfamily

%---------------------------------------------------------
% Kurzfassung
%---------------------------------------------------------
\chapter*{Kurzfassung} % Nicht im Inhaltsverzeichnis

Problemstellung

Ziel der Arbeit

Vorgehen und angewandte Methoden

Konkrete Ergebnisse der Arbeit, am besten mit quantitativen Angaben

\clearpage

%---------------------------------------------------------
% Abstract
%---------------------------------------------------------
\chapter*{Abstract} % Nicht im Inhaltsverzeichnis

English translation of the “Kurzfassung”.

\clearpage

%---------------------------------------------------------
% Nomenclature / Abbreviations
% (Requires running "makeindex main.nlo -s nomencl.ist -o main.nls")
%---------------------------------------------------------
% Uncomment or add your entries:
% \nomenclature[Abb]{Abb.}{Abbildung}
% \nomenclature[bzw]{bzw.}{beziehungsweise}
\nomenclature[etc]{etc.}{et cetera}
\nomenclature[f]{f.}{folgende Seite}
\nomenclature[ff]{ff.}{fortfolgende Seiten}
\nomenclature[vgl]{vgl.}{vergleiche}
\nomenclature[zB]{z. B.}{zum Beispiel}

% File types
\nomenclature[EMF]{EMF}{Enhanced Metafile}
\nomenclature[JPG]{JPG}{Joint Photographic Experts Group}
\nomenclature[KI]{KI}{Künstliche Intelligenz}
\nomenclature[PDF]{PDF}{Portable Document Format}
\nomenclature[PNG]{PNG}{Portable Network Graphics}

% Technical abbreviations
\nomenclature[ABS]{ABS}{Antiblockiersystem}
\nomenclature[ESC]{ESC}{Electronic Stability Control, Fahrdynamikregelung}

% Symbols
\nomenclature[a]{$a$}{Beschleunigung}
\nomenclature[F]{$F$}{Kraft}
\nomenclature[m]{$m$}{Masse}
\nomenclature[P]{$P$}{Leistung}
\nomenclature[U]{$U$}{Spannung}
\nomenclature[R]{$R$}{Widerstand}

%---------------------------------------------------------
% Main Chapters
%---------------------------------------------------------
\tableofcontents

\chapter{Grundlagen}
\label{cha:Grundlagen}

Zielgerichtete theoretische\cite{VIETH1999842} Grundlagen, sowohl fachliche, wie auch methodische.

\chapter{Umsetzung und Ergebnisse}
\label{cha:umsetzung}

Beschreibung der Umsetzung des zuvor gewählten Vorgehens, 
Verifikation anhand der Anforderungen etc.

%---------------------------------------------------------
% Example of a KOMA-Script addchap (unnumbered in ToC)
%---------------------------------------------------------
\addchap{A Nutzung von Künstliche Intelligenz basierten Werkzeugen}

Im Rahmen dieser Arbeit ...

\begin{table}[hbt]
    \centering
    \renewcommand{\arraystretch}{1.5}
    \captionabove{Liste der verwendeten KI-basierten Werkzeuge}
    \label{tab:anhang_uebersicht_KI_werkzeuge}
    \begin{tabular}{>{\raggedright\arraybackslash}p{0.3\linewidth}
                    >{\raggedright\arraybackslash}p{0.65\linewidth}}
        \textbf{Werkzeug} & \textbf{Beschreibung der Nutzung}\\
        \hline \hline
        ChatGPT &
        \begin{itemize}[noitemsep,topsep=0pt]
            \item Grundlagenrecherche …
            \item Suche nach Herstellern …
        \end{itemize} \\
        ChatPDF &
        \begin{itemize}[noitemsep,topsep=0pt]
            \item Recherche und Zusammenfassung von Studien …
        \end{itemize} \\
        DeepL &
        \begin{itemize}[noitemsep,topsep=0pt]
            \item Übersetzung von …
        \end{itemize} \\
        Tabnine AI coding assistant &
        \begin{itemize}[noitemsep,topsep=0pt]
            \item Aktiviertes Plugin …
        \end{itemize} \\
        \ldots & \ldots \\
        \hline
    \end{tabular}
\end{table}

%---------------------------------------------------------
% More Appendices, if needed
%---------------------------------------------------------
\addchap{B Ergänzungen}
% Additional info or sections ...

\addchap{C Details zu Laboraufbauten und Messergebnissen}
% Additional info ...

\addchap{D Zusatzinformationen zu verwendeter Software}
% Additional info ...

\addchap{E Datenblätter}
% Additional info ...

% Example of including PDF pages (pages 2-4)
\includepdf[pages={2-4}]{docs/EingebundenesPDF.pdf}

%---------------------------------------------------------
% Bibliography
%---------------------------------------------------------
\begingroup
\renewcommand{\bibfont}{\fontsize{13pt}{12pt}\selectfont}  % optionally larger font
\sloppy
\printbibliography
\endgroup

\end{document}
