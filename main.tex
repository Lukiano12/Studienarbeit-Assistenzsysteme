\documentclass[a4paper, 12pt]{article} % Artikel-Klasse

%---------------------------------------------------------
% Encoding, language, quotes
%---------------------------------------------------------
\usepackage[utf8]{inputenc}
\usepackage[ngerman]{babel}       % Deutsche Sprache und Silbentrennung
\usepackage{csquotes}             % Für korrekte Anführungszeichen

%---------------------------------------------------------
% Graphics & PDF
%---------------------------------------------------------
\usepackage{graphicx}
\usepackage{pdfpages}             % Einbinden von PDF-Seiten
\usepackage{caption}              % Verbesserte Bildunterschriften
\usepackage{subcaption}

%---------------------------------------------------------
% Math, units, spacing, etc.
%---------------------------------------------------------
\usepackage{siunitx}
\usepackage{setspace}
\usepackage{textgreek}

% Add float package for "H" float option
\usepackage{float}

%---------------------------------------------------------
% Other packages
%---------------------------------------------------------
\usepackage{ifthen}
\usepackage{acronym}
\PassOptionsToPackage{hyphens}{url} % URLs in Hyperlinks umbrechen
\usepackage[breaklinks=true]{hyperref} 
\usepackage{array}                % Bessere Tabellenformatierung
\usepackage{enumitem}             % Kontrolle über Listen-Layouts
\usepackage{nomencl}
\usepackage{scrlayer-scrpage}     % Header und Footer

% Adjust header and footer heights
\setlength{\headheight}{14.5pt}
\setlength{\footheight}{34.16666pt}

%---------------------------------------------------------
% Bibliography (biblatex mit Biber)
%---------------------------------------------------------
\usepackage[backend=biber, style=ieee]{biblatex}  
\addbibresource{literatur.bib}  

%---------------------------------------------------------
% Platzhalter
%---------------------------------------------------------
\newcommand{\titel}{Entwicklung eines mobilen Warnsystems zur Minimierung von Abbiegeunfällen zwischen LKW und Fußgänger:innen}
\newcommand{\untertitel}{}
\newcommand{\arbeit}{Studienarbeit T3100}
\newcommand{\studiengang}{Elektrotechnik}
\newcommand{\studienrichtung}{Fahrzeugelektronik}
\newcommand{\autor}{Luka Tadic}
\newcommand{\abgabe}{13.01.2025}
\newcommand{\bearbeitungszeitraum}{09.10.2024 - 13.01.2025}
\newcommand{\matrikelnr}{5726700}
\newcommand{\kurs}{TFE22-1}
\newcommand{\firma}{}
\newcommand{\betreuerfirma}{Prof. Dr. Ing. Tobias Frank}
\newcommand{\gutachterdhbw}{Prof. Dr. Ing. Tobias Frank}
\newcommand{\jahr}{2025}

%---------------------------------------------------------
% Header und Footer mit Linien
%---------------------------------------------------------
\clearpairofpagestyles         % Standard-Stile löschen

% Header with Logo Above the Line
\ohead{%
    \includegraphics[width=3cm]{images/DHBW_d_R_FN_46mm_4c}\\[1ex] % Logo above the line
    \rule[0.5\baselineskip]{\textwidth}{0.4pt} % Horizontal line below logo
}

% Footer with Line and Proper Alignment
\ifoot{%
    \rule[0.5\baselineskip]{\textwidth}{0.4pt}\\[1ex] % Horizontal line above footer
    \arbeit % Left-aligned title
}
\cfoot{%
    \rule[0.5\baselineskip]{\textwidth}{0.4pt}\\[1ex] % Line above center footer
    \autor % Center-aligned author name
}
\ofoot{%
    \rule[0.5\baselineskip]{\textwidth}{0.4pt}\\[1ex] % Line above right footer
    \thepage % Right-aligned page number
}

\pagestyle{scrheadings}        % Stil aktivieren

%---------------------------------------------------------
% Dokumentbeginn
%---------------------------------------------------------
\begin{document}

%---------------------------------------------------------
% Titelseite
%---------------------------------------------------------
\thispagestyle{empty}  % Kein Header oder Footer auf der Titelseite
\hypersetup{pageanchor=false}

\begin{titlepage}
\enlargethispage{4.0cm}
\sffamily  % Serifenlose Schrift für die Titelseite

\parbox{0.5\linewidth}{
    \begin{flushleft}
        % Optional: Firmenlogo
    \end{flushleft}
}
\parbox{0.5\linewidth}{
    \begin{flushright}
        \includegraphics[width=0.4\linewidth]{images/DHBW_d_R_FN_46mm_4c}\\[5ex]
    \end{flushright}
}

\begin{center}

{\fontsize{20.74pt}{24pt}\selectfont
\textbf{\titel}\\[1.5ex]}

{\fontsize{17pt}{20pt}\selectfont
\textbf{\arbeit}\\[2ex]}

{\fontsize{14pt}{17pt}\selectfont
Studiengang \studiengang\\[2ex]}

{\fontsize{12pt}{14pt}\selectfont
Studienrichtung \studienrichtung\\[1ex]
Duale Hochschule Baden-Württemberg Ravensburg, Campus Friedrichshafen\\[5ex]
von\\[1ex]
\autor\\[15ex]}

\end{center}

\begin{center}
{\fontsize{12pt}{14pt}\selectfont
\begin{tabular}{ll}
Abgabedatum:                    & \quad \abgabe \\  
Bearbeitungszeitraum:           & \quad \bearbeitungszeitraum \\  
Matrikelnummer:                 & \quad \matrikelnr \\  
Kurs:                           & \quad \kurs \\  
Dualer Partner:                 & \quad \firma \\ % entfällt bei Studienarbeit
Betreuerin / Betreuer:          & \quad \betreuerfirma \\  
Gutachterin / Gutachter:        & \quad \gutachterdhbw \\ [2ex]
\end{tabular}
}
\end{center}

\end{titlepage}

\clearpage

\pagestyle{scrheadings}  % Header und Footer nach Titelseite aktivieren
\hypersetup{pageanchor=true}

%---------------------------------------------------------
% Erklärung
%---------------------------------------------------------
\section*{Erklärung}

Ich versichere hiermit, dass ich meine \arbeit\ mit dem Thema:

\begin{quote}
    \textit{\titel}
\end{quote}

selbstständig verfasst und keine anderen als die angegebenen Quellen und Hilfsmittel benutzt habe.  
Ich versichere zudem, dass die eingereichte elektronische Fassung mit der gedruckten Fassung übereinstimmt.\\[6ex]

Friedrichshafen, den \today \\[1ex]
\rule[-0.2cm]{5cm}{0.5pt} \\  
\autor \\[10ex]

\rmfamily

\clearpage
%---------------------------------------------------------
% Abstract
%---------------------------------------------------------
\section*{Abstract}
English translation of the “Kurzfassung”.

\clearpage

%---------------------------------------------------------
% Inhaltsverzeichnis
%---------------------------------------------------------
\tableofcontents

\clearpage

%---------------------------------------------------------
% Hauptkapitel
%---------------------------------------------------------
\section{Grundlagen}
Zielgerichtete theoretische \cite{VIETH1999842}Grundlagen.

\clearpage

\section{Einleitung}
\begin{spacing}{1.8}  % Adjust line spacing
    \fontsize{14pt}{15pt}\selectfont  % Font size and line spacing

    



\begin{figure}[H]
    \includegraphics[width=0.5\linewidth]{images/DHBW_d_R_FN_46mm_4c.jpg}\\[1ex] 
    \centering
    \caption{}
    \label{ABBILDUNG 1}
\end{figure}

\end{spacing}

\clearpage

\section{Stand der Technik}


\begin{spacing}{1.8}  % Adjust line spacing
\fontsize{14pt}{15pt}\selectfont  % Font size and line spacing


\end{spacing}

\clearpage

\section{Idee}


\begin{spacing}{1.8}  % Adjust line spacing
\fontsize{14pt}{15pt}\selectfont  % Font size and line spacing


\end{spacing}

\clearpage

\subsection{Fahrradhelm}


\begin{spacing}{1.8}  % Adjust line spacing
\fontsize{14pt}{15pt}\selectfont  % Font size and line spacing


\end{spacing}

\clearpage

\section{Konzept}


\begin{spacing}{1.8}  % Adjust line spacing
\fontsize{14pt}{15pt}\selectfont  % Font size and line spacing


\end{spacing}

\clearpage

\section{Anforderungen und Zielgruppe}


\begin{spacing}{1.8}  % Adjust line spacing
\fontsize{14pt}{15pt}\selectfont  % Font size and line spacing


\end{spacing}

\clearpage

\section{Bluetooth}

sdafsafsafsasa\cite{wikipedia_bluetooth}

\begin{spacing}{1.8}  % Adjust line spacing
\fontsize{14pt}{15pt}\selectfont  % Font size and line spacing


\end{spacing}

\clearpage

\section{Entwicklung der App}


\begin{spacing}{1.8}  % Adjust line spacing
\fontsize{14pt}{15pt}\selectfont  % Font size and line spacing


\end{spacing}

\clearpage

\section{Eigenschaften}


\begin{spacing}{1.8}  % Adjust line spacing
\fontsize{14pt}{15pt}\selectfont  % Font size and line spacing


\end{spacing}

\clearpage

\section{Problematik}


\begin{spacing}{1.8}  % Adjust line spacing
\fontsize{14pt}{15pt}\selectfont  % Font size and line spacing


\end{spacing}

\clearpage

\section{Aussicht}


\begin{spacing}{1.8}  % Adjust line spacing
\fontsize{14pt}{15pt}\selectfont  % Font size and line spacing


\end{spacing}

\clearpage
%---------------------------------------------------------
% Bibliografie
%---------------------------------------------------------
\begingroup
\renewcommand{\bibfont}{\fontsize{13pt}{12pt}\selectfont}  
\sloppy
\printbibliography
\endgroup

\end{document}
